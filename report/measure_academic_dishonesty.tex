\documentclass{article}

\usepackage[margin=1in]{geometry}

\usepackage{graphicx}

\begin{document}
    \author{Kavi Gupta}
    \title{61A Exam Data Analysis: Measure of Academic Dishonesty}
    \maketitle

\section{General Strategy}
    The general strategy is to look at similarities between scores of students sitting next to each
        other against the similarities between scores of students sitting not next to each other.

    The result can be then used to estimate the number of students committing academic dishonesty
        using a model.

\section{Confounding Factors}
    There are many potential confounding factors, which we control for in different ways.
    \subsection{Grader Differences}
        Some graders grade differently from each other. Taking Midterm 1 problem 1.3, we can see that
            graders 5 and 8 had fairly typical behavior in terms of average scores given. However, we can
            also see that they have a very different profile of rubric items. In this case, what happened
            was that these two graders were giving rubric items 2, 3, and 4, which corresponded to each of
            the three parts of the problem being correct, rather than giving rubric item 1, which
            corresponded to the entire problem being correct. While this is irrelevant to the student, it
            does confound analyses.

        \begin{figure}[h!]
            \centering
            \includegraphics[width=\textwidth]{img/grader-comparison.png}
        \end{figure}

        Since grading ranges tend to be at least somewhat associated with location, this might lead to
            students near each other having more similar profiles, artificially. We control for graders
            by subtracting out from each student's individual problem scores and rubric items the mean
            given by a grader.

        Any student who had at least one problem that was either graded by a grader who had graded fewer
            than ten problems (not enough of a track record) or who had an abnormal pattern of rubric item
            grades\footnote{Unusualness of a grader is defined as
                $$u = \sum_k \left|\frac{x_k - \mu_k}{\sigma_k}\right|$$ where $x_k$ is the mean for the
                given grader of the $k$th rubric item, and $\mu_k, \sigma_k$ are the mean and standard
                deviation for that rubric item in general}.

        If we do not use individual rubric items, but instead overall question scores, we can use
            more similar formulas but discard less data as graders' question scores are more similar
            to each other.
    \subsection{Sequencing}
        The Gambler's Fallacy could possibly lead to a negative correlation between consecutively graded
            exams. To control for this, we simply ignore pairs of exams that are graded near each other in
            time.
    \subsection{Regional Differences}
        Individual rooms might have different rubric profiles, due to, for example, different TAs in
            different rooms. Additionally, individual rows or locations within rows might have
            different profiles themselves. To compensate for this, we compare the similarities of
            students sitting next to each other with the similarities of students sitting 2 seats
            away from each other.
    \subsection{Seating Chart Inaccuracies}
        Not yet fixed. Have the data, but would require a lot of data input because OCR is not particularly
            effective at recognizing handwriting.
\section{Measuring Similarity}
    \subsection{Rubric-Item Level Correlation}
        Using correlation between normalized scores as a measure of similarity, we can prove that
            there is a difference between the similarities in the set of adjacent pairs of students
            and the set of non-adjacent pairs of students for Midterm 1 and the final, but not for
            Midterm 2 (see controls above for more details). We can also see that the gambler's
            fallacy correction does not significantly impact it.

        \begin{figure}[h!]
            \centering
            \includegraphics[width=\textwidth]{img/matched-diff-rubric-correlation.png}
        \end{figure}
    \subsection{Absolute Difference in Overall Exam Score}
        An alternate means of similarity is far simpler. We look at the average difference
            between the scores in the two exams. For this measure of similarity, we do not need to
            remove extreme graders.

        Unfortunately, this does not seem to be granular enough to find cheaters.
        \begin{figure}[h!]
            \centering
            \includegraphics[width=\textwidth]{img/matched-diff-negative-abs-score-diff.png}
        \end{figure}
    \subsection{Question Level Correlation}
        A compromise between the power of the rubric-item level correlation and the simplicity of
            the absolute difference in exam score is question-level correlation. Unfortunately, it
            still lacks the power to see beyond the noise.
        \begin{figure}[h!]
            \centering
            \includegraphics[width=\textwidth]{img/matched-diff-question-correlation.png}
        \end{figure}
\section{Model For Academic Dishonesty}
    \subsection{Score-Independent, Binary Cheater, Random Cheat-Selection, Model}
        Assume that some fixed fraction $c$ of the students cheat by copying exactly $k$
            randomly selected points worth of material from a person sitting next to them (also
            selected randomly). Assume that non-cheating students get each point independently
            with probability $p$. Further assume that cheating students are located randomly.

        Using known exam data, we can approximate $p$ simply by taking the average score divided
            by the total number of points.

        Unfortunately, the grades this model produces do not approximate the true distribution
            of grades, as the graph below demonstrates.

    \subsection{Question-Independent, Binary Cheater, Random Cheat-Selection Model}
        This model is equivalent to the one above, but uses point chunks representing each
            question, which are assumed to be independent and take the sizes of each of the
            questions. A certain number of questions rather than points are copied. Each question
            is assigned a score based on a normal distribution from the mean and variance of
            the real data for that question.

        Unfortunately, this strategy also produces a distribution of grades dissimilar to the
            original data.

        \begin{figure}[h!]
            \centering
            \includegraphics[width=\textwidth]{img/independents-not-working.png}
        \end{figure}
    \subsection{Random Seating, Binary Cheater, Random Cheat-Selection Model}
        This is similar to above, but the initial non-cheater distribution is created by
            shuffling the students into random locations. The pitfall of this method is that
            some information about the cheating is encoded in this distribution.

        However, if we assume that the amount of cheating is fairly low, it should be a fairly
            accurate model overall.

\end{document}
