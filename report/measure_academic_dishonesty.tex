\documentclass{article}

\usepackage[margin=1in]{geometry}

\usepackage{graphicx}

\begin{document}
    \author{Kavi Gupta}
    \title{61A Exam Data Analysis: Measure of Academic Dishonesty}
    \maketitle

\section{General Strategy}
    The general strategy is to look at similarities between scores of students sitting next to each
        other against the similarities between scores of students sitting not next to each other.
    
    The result can be then used to estimate the number of students committing academic dishonesty
        using a model.

\section{Confounding Factors}
    There are many potential confounding factors, which we control for in different ways.
    \subsection{Grader Differences}
        Some graders grade differently from each other. Taking Midterm 1 problem 1.3, we can see that
            graders 5 and 8 had fairly typical behavior in terms of average scores given. However, we can
            also see that they have a very different profile of rubric items. In this case, what happened
            was that these two graders were giving rubric items 2, 3, and 4, which corresponded to each of
            the three parts of the problem being correct, rather than giving rubric item 1, which
            corresponded to the entire problem being correct. While this is irrelevant to the student, it
            does confound analyses.

        \begin{figure}[h!]
            \centering
            \includegraphics[width=\textwidth]{img/grader-comparison.png}
        \end{figure}

        Since grading ranges tend to be at least somewhat associated with location, this might lead to
            students near each other having more similar profiles, artificially. We control for graders
            by subtracting out from each student's individual problem scores and rubric items the mean
            given by a grader.
            
        Any student who had at least one problem that was either graded by a grader who had graded fewer
            than ten problems (not enough of a track record) or who had an abnormal pattern of rubric item
            grades\footnote{Unusualness of a grader is defined as
                $$u = \sum_k \left|\frac{x_k - \mu_k}{\sigma_k}\right|$$ where $x_k$ is the mean for the 
                given grader of the $k$th rubric item, and $\mu_k, \sigma_k$ are the mean and standard
                deviation for that rubric item in general}.
    \subsection{Sequencing}
        The Gambler's Fallacy could possibly lead to a negative correlation between consecutively graded
            exams. To control for this, we simply ignore pairs of exams that are graded near each other in
            time.
    \subsection{Room}
        Individual rooms might have different rubric profiles, due to, for example, different TAs in
            different rooms. It looks like no particular room has that much of a problem, so for now
            rooms are not controlled for except to not take pairs from different rooms (there is a
            small amount of variation)
        \begin{figure}[h!]
            \centering
            \includegraphics[width=\textwidth]{img/room-comparison.png}
        \end{figure}

\end{document}
